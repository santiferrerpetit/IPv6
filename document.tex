\documentclass[]{article}
\usepackage{bytefield}
\usepackage{geometry}

\geometry{a4paper, margin=1.4in}

% Título y autores
\title{Protocolo IPv6}
\author{Santiago Ferrer Petit, Manuel García de la Vega}

\begin{document}
	
	\maketitle
	
	\section{\textbf{IPv6 SLAAC and EUI-64 Basics}}
	
	\subsection{\textbf{Configuración del router en IPv6}}
	
	Al buscar la LLA en la configuración de la PC0, encontramos que su dirección IPv6 es \texttt{FE80::2E0:F9FF:FE98:8A07}.  
	La dirección MAC es \texttt{00E0:F998:8A07}.
	
	El método EUI-64 (Extended Unique Identifier - 64 bits) permite generar automáticamente la parte de host de una dirección IPv6 a partir de la dirección MAC de la tarjeta de red del dispositivo.  
	Este método se usa cuando se emplea SLAAC (Stateless Address Autoconfiguration) para configurar direcciones IPv6 sin un servidor DHCPv6.
	
	El proceso de conversión de una dirección MAC a un identificador EUI-64 sigue estos pasos:
	
	\begin{itemize}
		\item Se toma la dirección MAC de 48 bits del dispositivo.
		\item Se inserta el valor hexadecimal \texttt{0xFFFE} en el medio de la dirección MAC para extenderla a 64 bits.
		\item Se modifica el séptimo bit del primer byte de la dirección MAC (Universal/Local bit) para indicar que la dirección ha sido modificada mediante EUI-64.
	\end{itemize}
	
	\textbf{Cambio de Static a Automatic:}
	
	Al cambiar la configuración de IPv6 de Static a Automatic, la PC activa la función SLAAC (Stateless Address Autoconfiguration).
	
	SLAAC permite que un dispositivo obtenga automáticamente una dirección IPv6 sin necesidad de un servidor DHCPv6.  
	Para esto, la PC necesita encontrar un router en la red que pueda proporcionarle la información necesaria.
	
	\subsection{\textbf{Análisis del mensaje RS (Router Solicitation)}}
	
	El mensaje Router Solicitation (RS) es una solicitud que la PC envía a todos los routers en la red para obtener información de configuración.\\
	
	\begin{bytefield}[bitwidth=1.2em]{32}
		\bitbox{8}{\small TYPE: 0x85} & 
		\bitbox{8}{\small CODE: 0x00} & 
		\bitbox{16}{\small CHECKSUM: 0x0000} \\
		\wordbox{1}{RESERVED} \\
		\wordbox{1}{OPTION}
	\end{bytefield}
	
	Explicación de los campos:
	
	\begin{itemize}
		\item \textbf{TYPE (0x85)}: Indica que es un mensaje Router Solicitation.
		\item \textbf{CODE (0x00)}: Siempre es 0 en este tipo de mensaje.
		\item \textbf{CHECKSUM}: Usado para la verificación de integridad.
		\item \textbf{RESERVED}: Campo reservado, debe ser cero.
		\item \textbf{OPTION}: Puede incluir la dirección MAC del nodo que envía el mensaje.
	\end{itemize}
	
	\subsection{\textbf{Análisis del mensaje RA (Router Advertisement)}}
	
	Cuando el router responde, envía un mensaje RA (Router Advertisement) con los siguientes datos:\\
	
	\begin{bytefield}[bitwidth=1.2em]{32}
		\bitbox{8}{\small TYPE: 0x86} & 
		\bitbox{8}{\small CODE: 0x00} & 
		\bitbox{16}{\small CHECKSUM: 0x0000} \\
		\bitbox{8}{\small Hop Limit: 0x40} & 
		\bitbox{8}{\small RESERVED} & 
		\bitbox{16}{\small Router Lifetime: 0x0708} \\
		\wordbox{2}{Reachable Time: 0x00000000} \\
		\wordbox{2}{Retrans Timer: 0x00000000}
	\end{bytefield}
	
	Explicación de los campos:
	
	\begin{itemize}
		\item \textbf{TYPE (0x86)}: Indica que es un mensaje Router Advertisement.
		\item \textbf{CODE (0x00)}: Siempre es 0 en este mensaje.
		\item \textbf{CHECKSUM}: Verificación de integridad del mensaje.
		\item \textbf{Hop Limit (0x40)}: Valor recomendado para el campo TTL de los paquetes enviados.
		\item \textbf{RESERVED}: Contiene banderas como "M" (Managed) y "O" (Other), usadas en DHCPv6.
		\item \textbf{Router Lifetime (0x0708)}: Tiempo en el que el router puede ser la puerta de enlace predeterminada.
		\item \textbf{Reachable Time (0x00000000)}: Tiempo en que un nodo es considerado alcanzable antes de verificar nuevamente.
		\item \textbf{Retrans Timer (0x00000000)}: Intervalo entre retransmisiones de mensajes.
	\end{itemize}
	
	\subsection{\textbf{Verificación de la dirección IPv6 asignada (GUA)}}
	
	Después de recibir el mensaje RA, la PC crea su dirección Global Unicast Address (GUA).  
	Para ello, combina:
	
	\begin{itemize}
		\item El prefijo de red recibido en el mensaje RA.
		\item Un identificador de interfaz, que puede generarse con EUI-64 o de manera aleatoria.
	\end{itemize}
	
	\section{\textbf{Neighbor Discovery}}
	
	\subsection{\textbf{Local Delivery}}
	
	\subsection{\textbf{Non Local Delivery}}
	
	\section{\textbf{Conclusiones}}
	
\end{document}

